\begin{table}[htbp]\centering
\small
\caption{Average Montonicity for Initial Assessment of Mental Health Court (Under 25)}\label{tab:avgmono}
\begin{center}
\begin{threeparttable}
\begin{tabular}{p{.3\textwidth}*{5}{c}}
\toprule
                    &\multicolumn{1}{c}{Male}&\multicolumn{1}{c}{Female}&\multicolumn{1}{c}{Black}&\multicolumn{1}{c}{White}&\multicolumn{1}{c}{Hispanic}\\\cmidrule(lr){2-2}\cmidrule(lr){3-3}\cmidrule(lr){4-4}\cmidrule(lr){5-5}\cmidrule(lr){6-6}
                    &\multicolumn{1}{c}{(1)}   &\multicolumn{1}{c}{(2)}   &\multicolumn{1}{c}{(3)}   &\multicolumn{1}{c}{(4)}   &\multicolumn{1}{c}{(5)}   \\
\midrule
{\hangindent=2emZ: Clinician's Leave-Out Mean Mental Health Score}&       0.523***&       0.769***&       0.579***&       0.585***&       0.442***\\
                    &     (0.096)   &     (0.145)   &     (0.119)   &     (0.109)   &     (0.090)   \\
\midrule
Observations        &       6,717   &       2,815   &       2,468   &       6,946   &       3,411   \\
Time Fixed Effects  &         Yes   &         Yes   &         Yes   &         Yes   &         Yes   \\
Controls            &         Yes   &         Yes   &         Yes   &         Yes   &         Yes   \\
\bottomrule
\end{tabular}
\tiny
This table reports the first stage results by subsamples as listed in the column headers, which serves as informal evidence of average monotonicity if the estimate is significant across all subsamples. * p$<$0.10, ** p$<$0.05, *** p$<$0.01
\end{threeparttable}
\end{center}
\end{table}
